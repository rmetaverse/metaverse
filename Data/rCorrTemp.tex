\documentclass[]{article}
\usepackage{lmodern}
\usepackage{amssymb,amsmath}
\usepackage{ifxetex,ifluatex}
\usepackage{fixltx2e} % provides \textsubscript
\ifnum 0\ifxetex 1\fi\ifluatex 1\fi=0 % if pdftex
  \usepackage[T1]{fontenc}
  \usepackage[utf8]{inputenc}
\else % if luatex or xelatex
  \ifxetex
    \usepackage{mathspec}
  \else
    \usepackage{fontspec}
  \fi
  \defaultfontfeatures{Ligatures=TeX,Scale=MatchLowercase}
\fi
% use upquote if available, for straight quotes in verbatim environments
\IfFileExists{upquote.sty}{\usepackage{upquote}}{}
% use microtype if available
\IfFileExists{microtype.sty}{%
\usepackage{microtype}
\UseMicrotypeSet[protrusion]{basicmath} % disable protrusion for tt fonts
}{}
\usepackage[margin=1in]{geometry}
\usepackage{hyperref}
\hypersetup{unicode=true,
            pdftitle={Standard Meta Analysis Template for Correlations},
            pdfborder={0 0 0},
            breaklinks=true}
\urlstyle{same}  % don't use monospace font for urls
\usepackage{color}
\usepackage{fancyvrb}
\newcommand{\VerbBar}{|}
\newcommand{\VERB}{\Verb[commandchars=\\\{\}]}
\DefineVerbatimEnvironment{Highlighting}{Verbatim}{commandchars=\\\{\}}
% Add ',fontsize=\small' for more characters per line
\usepackage{framed}
\definecolor{shadecolor}{RGB}{248,248,248}
\newenvironment{Shaded}{\begin{snugshade}}{\end{snugshade}}
\newcommand{\KeywordTok}[1]{\textcolor[rgb]{0.13,0.29,0.53}{\textbf{#1}}}
\newcommand{\DataTypeTok}[1]{\textcolor[rgb]{0.13,0.29,0.53}{#1}}
\newcommand{\DecValTok}[1]{\textcolor[rgb]{0.00,0.00,0.81}{#1}}
\newcommand{\BaseNTok}[1]{\textcolor[rgb]{0.00,0.00,0.81}{#1}}
\newcommand{\FloatTok}[1]{\textcolor[rgb]{0.00,0.00,0.81}{#1}}
\newcommand{\ConstantTok}[1]{\textcolor[rgb]{0.00,0.00,0.00}{#1}}
\newcommand{\CharTok}[1]{\textcolor[rgb]{0.31,0.60,0.02}{#1}}
\newcommand{\SpecialCharTok}[1]{\textcolor[rgb]{0.00,0.00,0.00}{#1}}
\newcommand{\StringTok}[1]{\textcolor[rgb]{0.31,0.60,0.02}{#1}}
\newcommand{\VerbatimStringTok}[1]{\textcolor[rgb]{0.31,0.60,0.02}{#1}}
\newcommand{\SpecialStringTok}[1]{\textcolor[rgb]{0.31,0.60,0.02}{#1}}
\newcommand{\ImportTok}[1]{#1}
\newcommand{\CommentTok}[1]{\textcolor[rgb]{0.56,0.35,0.01}{\textit{#1}}}
\newcommand{\DocumentationTok}[1]{\textcolor[rgb]{0.56,0.35,0.01}{\textbf{\textit{#1}}}}
\newcommand{\AnnotationTok}[1]{\textcolor[rgb]{0.56,0.35,0.01}{\textbf{\textit{#1}}}}
\newcommand{\CommentVarTok}[1]{\textcolor[rgb]{0.56,0.35,0.01}{\textbf{\textit{#1}}}}
\newcommand{\OtherTok}[1]{\textcolor[rgb]{0.56,0.35,0.01}{#1}}
\newcommand{\FunctionTok}[1]{\textcolor[rgb]{0.00,0.00,0.00}{#1}}
\newcommand{\VariableTok}[1]{\textcolor[rgb]{0.00,0.00,0.00}{#1}}
\newcommand{\ControlFlowTok}[1]{\textcolor[rgb]{0.13,0.29,0.53}{\textbf{#1}}}
\newcommand{\OperatorTok}[1]{\textcolor[rgb]{0.81,0.36,0.00}{\textbf{#1}}}
\newcommand{\BuiltInTok}[1]{#1}
\newcommand{\ExtensionTok}[1]{#1}
\newcommand{\PreprocessorTok}[1]{\textcolor[rgb]{0.56,0.35,0.01}{\textit{#1}}}
\newcommand{\AttributeTok}[1]{\textcolor[rgb]{0.77,0.63,0.00}{#1}}
\newcommand{\RegionMarkerTok}[1]{#1}
\newcommand{\InformationTok}[1]{\textcolor[rgb]{0.56,0.35,0.01}{\textbf{\textit{#1}}}}
\newcommand{\WarningTok}[1]{\textcolor[rgb]{0.56,0.35,0.01}{\textbf{\textit{#1}}}}
\newcommand{\AlertTok}[1]{\textcolor[rgb]{0.94,0.16,0.16}{#1}}
\newcommand{\ErrorTok}[1]{\textcolor[rgb]{0.64,0.00,0.00}{\textbf{#1}}}
\newcommand{\NormalTok}[1]{#1}
\usepackage{longtable,booktabs}
\usepackage{graphicx,grffile}
\makeatletter
\def\maxwidth{\ifdim\Gin@nat@width>\linewidth\linewidth\else\Gin@nat@width\fi}
\def\maxheight{\ifdim\Gin@nat@height>\textheight\textheight\else\Gin@nat@height\fi}
\makeatother
% Scale images if necessary, so that they will not overflow the page
% margins by default, and it is still possible to overwrite the defaults
% using explicit options in \includegraphics[width, height, ...]{}
\setkeys{Gin}{width=\maxwidth,height=\maxheight,keepaspectratio}
\IfFileExists{parskip.sty}{%
\usepackage{parskip}
}{% else
\setlength{\parindent}{0pt}
\setlength{\parskip}{6pt plus 2pt minus 1pt}
}
\setlength{\emergencystretch}{3em}  % prevent overfull lines
\providecommand{\tightlist}{%
  \setlength{\itemsep}{0pt}\setlength{\parskip}{0pt}}
\setcounter{secnumdepth}{0}
% Redefines (sub)paragraphs to behave more like sections
\ifx\paragraph\undefined\else
\let\oldparagraph\paragraph
\renewcommand{\paragraph}[1]{\oldparagraph{#1}\mbox{}}
\fi
\ifx\subparagraph\undefined\else
\let\oldsubparagraph\subparagraph
\renewcommand{\subparagraph}[1]{\oldsubparagraph{#1}\mbox{}}
\fi

%%% Use protect on footnotes to avoid problems with footnotes in titles
\let\rmarkdownfootnote\footnote%
\def\footnote{\protect\rmarkdownfootnote}

%%% Change title format to be more compact
\usepackage{titling}

% Create subtitle command for use in maketitle
\providecommand{\subtitle}[1]{
  \posttitle{
    \begin{center}\large#1\end{center}
    }
}

\setlength{\droptitle}{-2em}

  \title{Standard Meta Analysis Template for Correlations}
    \pretitle{\vspace{\droptitle}\centering\huge}
  \posttitle{\par}
    \author{}
    \preauthor{}\postauthor{}
    \date{}
    \predate{}\postdate{}
  

\begin{document}
\maketitle

\subsection{About the analysis}\label{about-the-analysis}

This file documents the analyses conducted for LastName (2019)
\emph{TitleGoesHere}

Analyses were conducted using the file dat.molloy2014.csv, to examine
the relationship between Intention and Behaviour. Correlation
coefficents were transformed to Fisher Z correlation coefficents for
analysis and backtransformed for reporting.

It is expected that datafiles for analysis are provided in the format
outlined here:
\url{https://osf.io/6bk7b/wiki/Correlation\%20Data\%20Format/}

\subsection{Data importing and effect size
calculation}\label{data-importing-and-effect-size-calculation}

Read dat.molloy2014.csv into R

\begin{Shaded}
\begin{Highlighting}[]
\CommentTok{#Read your datafile into R}
\NormalTok{mydata<-}\KeywordTok{read.csv}\NormalTok{(filename)}
\end{Highlighting}
\end{Shaded}

\begin{longtable}[]{@{}rlrrrllllrr@{}}
\caption{First few rows of the imported data}\tabularnewline
\toprule
X & authors & year & ni & ri & controls & design & a\_measure &
c\_measure & meanage & quality\tabularnewline
\midrule
\endfirsthead
\toprule
X & authors & year & ni & ri & controls & design & a\_measure &
c\_measure & meanage & quality\tabularnewline
\midrule
\endhead
1 & Axelsson et al. & 2009 & 109 & 0.187 & none & cross-sectional &
self-report & other & 22.00 & 1\tabularnewline
2 & Axelsson et al. & 2011 & 749 & 0.162 & none & cross-sectional &
self-report & NEO & 53.59 & 1\tabularnewline
\bottomrule
\end{longtable}

Calculate r-to-z transformed correlations and corresponding sampling
variances

\begin{Shaded}
\begin{Highlighting}[]
\KeywordTok{library}\NormalTok{(metafor)}
\CommentTok{# calculate r-to-z transformed correlations and corresponding sampling variances}
\NormalTok{dat <-}\StringTok{ }\KeywordTok{escalc}\NormalTok{(}\DataTypeTok{ri=}\NormalTok{ri, }\DataTypeTok{ni=}\NormalTok{ni, }\DataTypeTok{measure=}\NormalTok{specified.measure, }\DataTypeTok{data=}\NormalTok{mydata, }\DataTypeTok{append=}\OtherTok{TRUE}\NormalTok{)}
\end{Highlighting}
\end{Shaded}

\begin{longtable}[]{@{}rlrrrllllrrrr@{}}
\caption{First few rows of the effect size table}\tabularnewline
\toprule
X & authors & year & ni & ri & controls & design & a\_measure &
c\_measure & meanage & quality & yi & vi\tabularnewline
\midrule
\endfirsthead
\toprule
X & authors & year & ni & ri & controls & design & a\_measure &
c\_measure & meanage & quality & yi & vi\tabularnewline
\midrule
\endhead
1 & Axelsson et al. & 2009 & 109 & 0.187 & none & cross-sectional &
self-report & other & 22.00 & 1 & 0.1892266 & 0.0094340\tabularnewline
2 & Axelsson et al. & 2011 & 749 & 0.162 & none & cross-sectional &
self-report & NEO & 53.59 & 1 & 0.1634399 & 0.0013405\tabularnewline
\bottomrule
\end{longtable}

\subsection{Conducting the
meta-analyis}\label{conducting-the-meta-analyis}

\begin{Shaded}
\begin{Highlighting}[]
\CommentTok{# Run the random effect meta-analysis}
\NormalTok{res <-}\StringTok{ }\KeywordTok{rma}\NormalTok{(yi, vi, }\DataTypeTok{data=}\NormalTok{dat)}

\CommentTok{# Back transform z to r correlations for reporting if ZCOR was used}
\NormalTok{transformed<-}\KeywordTok{ifelse}\NormalTok{(specified.measure}\OperatorTok{==}\StringTok{"ZCOR"}\NormalTok{, }
\NormalTok{                    res_back<-}\KeywordTok{predict}\NormalTok{(res, }\DataTypeTok{tranf=}\NormalTok{transf.ztor), }
\NormalTok{                    res_back<-}\KeywordTok{predict}\NormalTok{(res))}

\CommentTok{# Create a table of information from the res model}
\NormalTok{res.table<-}\KeywordTok{cbind.data.frame}\NormalTok{(res_back}\OperatorTok{$}\NormalTok{pred, res_back}\OperatorTok{$}\NormalTok{se, }
\NormalTok{                            res}\OperatorTok{$}\NormalTok{pval, res_back}\OperatorTok{$}\NormalTok{ci.lb, res_back}\OperatorTok{$}\NormalTok{ci.ub,  res}\OperatorTok{$}\NormalTok{k)}
\KeywordTok{colnames}\NormalTok{(res.table)<-}\KeywordTok{c}\NormalTok{(}\StringTok{"r"}\NormalTok{, }\StringTok{"se"}\NormalTok{, }\StringTok{"p"}\NormalTok{, }\StringTok{"CI.LB"}\NormalTok{, }\StringTok{"CI.UB"}\NormalTok{, }\StringTok{"k"}\NormalTok{)}
\KeywordTok{row.names}\NormalTok{(res.table)<-}\KeywordTok{paste}\NormalTok{(X,}\StringTok{"-"}\NormalTok{,Y)}

\CommentTok{# Create a table of heterogeneity information}
\NormalTok{het.table<-}\KeywordTok{cbind.data.frame}\NormalTok{(res}\OperatorTok{$}\NormalTok{tau2, res}\OperatorTok{$}\NormalTok{se.tau2, res}\OperatorTok{$}\NormalTok{QE, }
\NormalTok{                            res}\OperatorTok{$}\NormalTok{QEp, res}\OperatorTok{$}\NormalTok{I2, res_back}\OperatorTok{$}\NormalTok{cr.lb, res_back}\OperatorTok{$}\NormalTok{cr.ub)}
\KeywordTok{colnames}\NormalTok{(het.table)<-}\KeywordTok{c}\NormalTok{(}\StringTok{"tau2"}\NormalTok{, }\StringTok{"se.tau2"}\NormalTok{, }\StringTok{"Q"}\NormalTok{, }\StringTok{"p"}\NormalTok{, }\StringTok{"I2"}\NormalTok{, }\StringTok{"cr.lb"}\NormalTok{, }\StringTok{"cr.ub"}\NormalTok{)}
\KeywordTok{row.names}\NormalTok{(het.table)<-}\KeywordTok{paste}\NormalTok{(X,}\StringTok{"-"}\NormalTok{,Y)}
\end{Highlighting}
\end{Shaded}

\subsection{Results}\label{results}

\textbf{Summary}

This analysis is based on 16 studies that evaluated the relationship
between Intention and Behaviour. Correlation coefficents were
transformed to Fisher Z correlation coefficents for analysis and
backtransformed for reporting.

\includegraphics{rCorrTemp_files/figure-latex/unnamed-chunk-6-1.pdf}

\textbf{What is the strength of the association between Intention and
Behaviour?}

A random effects meta-analysis was conducted (k=16) to explore the
association between Intention and Behaviour. The average correlation
between these variables is r\textsubscript{+}=0.15, (p=0, 95\% CI
{[}0.09, 0.21{]}). See table below.

\begin{longtable}[]{@{}lrrrrrr@{}}
\caption{Meta-Analysis Summary Table}\tabularnewline
\toprule
& r & se & p & CI Lower & CI Upper & k\tabularnewline
\midrule
\endfirsthead
\toprule
& r & se & p & CI Lower & CI Upper & k\tabularnewline
\midrule
\endhead
Intention - Behaviour & 0.15 & 0.032 & 0 & 0.088 & 0.212 &
16\tabularnewline
\bottomrule
\end{longtable}

\textbf{Does the strength of the association vary across studies?}

A Cochran's Q test was conducted to examine whether variations in the
observed correlation are likely to be attributable soley to sampling
error (Q\textsubscript{(df=15)}=38.16, p=\textless{}.001). The variation
in the correlation is greater than would be expected from sampling error
alone. It appears that the true correlation varies betweeen studies.

The I\textsuperscript{2} statistics indicates the \emph{proportion} of
variance in the observed effect attributable to sampling error. In this
instance, the I\textsuperscript{2} = 61.73\%.

Note, this statistic is not an absolute measure of heterogeneity
(although it is often interpreted as such). We strongly advise against
using rules of thumb such as ``small'', ``medium'' or ``large'' when
interpreting I\textsuperscript{2} values. Instead, researchers
increasingly argue that the information provided credibility or
prediction intervals is more useful in understanding the heterogeneity
of true effect sizes in meta-analysis. In this instance the 95\%
credibility intervals are -0.04,0.34. That is, it is estimated that 95\%
of true correlations fall between r=-0.04 and r=0.34.

Heterogeneity statistics are summarised below

\begin{longtable}[]{@{}lrrrrrrr@{}}
\caption{Heterogeneity Summary Table}\tabularnewline
\toprule
& \(\tau\)\textsuperscript{2} & se \(\tau\)\textsuperscript{2} & Q & p &
I\textsuperscript{2} & Credibility Lower & Credibility
Upper\tabularnewline
\midrule
\endfirsthead
\toprule
& \(\tau\)\textsuperscript{2} & se \(\tau\)\textsuperscript{2} & Q & p &
I\textsuperscript{2} & Credibility Lower & Credibility
Upper\tabularnewline
\midrule
\endhead
Intention - Behaviour & 0.008 & 0.006 & 38.16 & 0.001 & 61.732 & -0.037
& 0.337\tabularnewline
\bottomrule
\end{longtable}

\newpage

\section{Notes}\label{notes}

\texttt{r\ \ sessionInfo()}

\texttt{\#\#\ R\ version\ 3.5.3\ (2019-03-11)\ \ \#\#\ Platform:\ x86\_64-apple-darwin15.6.0\ (64-bit)\ \ \#\#\ Running\ under:\ macOS\ Mojave\ 10.14.3\ \ \#\#\ \ \ \#\#\ Matrix\ products:\ default\ \ \#\#\ BLAS:\ /Library/Frameworks/R.framework/Versions/3.5/Resources/lib/libRblas.0.dylib\ \ \#\#\ LAPACK:\ /Library/Frameworks/R.framework/Versions/3.5/Resources/lib/libRlapack.dylib\ \ \#\#\ \ \ \#\#\ locale:\ \ \#\#\ {[}1{]}\ en\_US.UTF-8/en\_US.UTF-8/en\_US.UTF-8/C/en\_US.UTF-8/en\_US.UTF-8\ \ \#\#\ \ \ \#\#\ attached\ base\ packages:\ \ \#\#\ {[}1{]}\ stats\ \ \ \ \ graphics\ \ grDevices\ utils\ \ \ \ \ datasets\ \ methods\ \ \ base\ \ \ \ \ \ \ \#\#\ \ \ \#\#\ other\ attached\ packages:\ \ \#\#\ {[}1{]}\ knitr\_1.22\ \ \ \ metafor\_2.0-0\ Matrix\_1.2-17\ \ \#\#\ \ \ \#\#\ loaded\ via\ a\ namespace\ (and\ not\ attached):\ \ \#\#\ \ {[}1{]}\ Rcpp\_1.0.1\ \ \ \ \ \ lattice\_0.20-38\ digest\_0.6.18\ \ \ grid\_3.5.3\ \ \ \ \ \ \ \#\#\ \ {[}5{]}\ nlme\_3.1-137\ \ \ \ magrittr\_1.5\ \ \ \ evaluate\_0.13\ \ \ highr\_0.8\ \ \ \ \ \ \ \ \#\#\ \ {[}9{]}\ stringi\_1.4.3\ \ \ rmarkdown\_1.12\ \ tools\_3.5.3\ \ \ \ \ stringr\_1.4.0\ \ \ \ \#\#\ {[}13{]}\ xfun\_0.6\ \ \ \ \ \ \ \ yaml\_2.2.0\ \ \ \ \ \ compiler\_3.5.3\ \ htmltools\_0.3.6}


\end{document}
